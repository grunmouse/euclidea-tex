\section{Теоремы}
\subsection{Теорема о срединном перпендикуляре:} Каждая точка срединного перпендикуляра равноудалена от концов отрезка.

Пусть $AB$ - отрезок, $O$ - его середина, $K$ - произвольная точка, принадлежащая срединному перпендикуляру.

$\angle AOK = 90°$ по определению перпендикуляра.

$\triangle AOK$ - прямоугольный.

$\triangle BOK$ - прямоугольный.

$AO = BO$ - по условию середины отрезка.

$OK$ - общий катет.

$\triangle AOK = \triangle BOK$ по двум катетам.

$AK = BK$
