\section{$\alpha $}
\subsection{Угол $60°$}
\stealgraphics{task-01/01.pdf}
\subsection{Срединный перпендикуляр}
\stealgraphics{task-02/01.pdf}
\paragraph{Теорема о срединном перпендикуляре:} Каждая точка срединного перпендикуляра равноудалена от концов отрезка.
Следовательно, прямая, проходящая через две такие точки является срединным перпендикуляром.
\subsection{Середина отрезка}
\stealgraphics{task-03/01.pdf}
Середина отрезка находится в точке его пересечения со срединным перпендикуляром.
\subsection{Окружность, вписанная в квадрат}
\subsection{Ромб, ваисанный в пярмоугольник}
\subsection{Центр окружности}
\subsection{Квадрат, вписанный в окружность}